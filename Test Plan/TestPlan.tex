\documentclass[12pt]{article}
\usepackage{xcolor} % for different colour comments


%\usepackage{refcheck}
\newif\ifcomments\commentstrue

\ifcomments
\newcommand{\authornote}[3]{\textcolor{#1}{[#3 ---#2]}}
\newcommand{\todo}[1]{\textcolor{red}{[TODO: #1]}}
\else
\newcommand{\authornote}[3]{}
\newcommand{\todo}[1]{}
\fi

\newcommand{\wss}[1]{\authornote{magenta}{SS}{#1}}
\newcommand{\ds}[1]{\authornote{blue}{DS}{#1}}


\usepackage{hyperref}
\usepackage{graphicx}
\usepackage{verbatim} 
\graphicspath{ {Images/} }


\begin{document}

\title{Smart Waiter Test Plan} 
\author{Meraj Patel (patelmu2) \\ Pavneet Jauhal (jauhalps)\\ Shan Perera (pererali)}
\date{\today}
\maketitle
\ds{Missing Student Numbers.}

\ds{Missing the ability to add comments like these, I had to add it.}
\pagebreak

\tableofcontents

\listoftables

\section*{Revision History}
\includegraphics[width=\textwidth,height=\textheight,keepaspectratio]{revision.png}


%\begin{table}[H]
%\section*{Revision History}
%\begin{tabular}{|c|c|}
%\hline
%\textbf{Date}  & \textbf{Comments} \\ \hline
%October 30, 2015 &  first draft. \\ 
%\hline
%\end{tabular}
%\caption{Revision History Table}
%\end{table}

\pagebreak
%%%%%%%%%%%%%%%%%%%%%%%%
%
%	1.) General Information 
%
%%%%%%%%%%%%%%%%%%%%%%%%

\section{General Information}
The following section provides an overview of the test plan 
for Smart-Waiter.
 This section explains the purpose of this document, the scope of the system, and an overview of the following sections

%1.1 Purpose
\subsection{Purpose}
The purpose of this document is to describe  various test cases and procedures to evaluate functionality of Smart Waiter.
This document is indented to be used as a reference for all future testing. 

%1.2 Scope
\subsection{Scope}
This test plan is used to evaluate Smart Waiter functionality. Various plans are described in detail to test expected functional and non function requirements.  

\subsection{Overview of Document }
The following sections provide more detail about types of test that will be used. Information about the testing process is provided, and the software specifications
that were discussed in the SRS document are stated. Testing processes is split up between system test for POC and the final demonstration. 

\ds{Acronyms should be defined somewhere.}

%%%%%%%%%%%%%%%%%%%%%%%%
%
%	2.) Plan
%
%%%%%%%%%%%%%%%%%%%%%%%%

\section{Plan}
This section provides a description of the software that is being tested, the team that will
perform the testing, the milestones for the testing phase, and the budget allocated to the testing. 

%2.1 Software Description
\subsection{Software Description}
The software being tested is for Smart Waiter Solutions. The project aims to provide a solution that will allow users to order and pay through a mobile application at restaurants. 

%2.2 Test Team
\subsection{Test Team} 
The team that will execute the test cases, write and review Smart Waiter consists of:

\begin{itemize}
 \item Pavneet Jauhal
 \item Shan Perera
 \item Meraj Patel
 \item Dr.\ Rong Zheng
\end{itemize}  

%2.3 Milestones
\subsection{Milestones}

%2.3.1 Location
\subsubsection{Location}
The location where the testing will be performed is Hamilton Ontario. Specifically, we will be performing the testing in McMaster University. 


%2.3.1 Dates and Deadlines
\subsubsection{Dates and Deadlines}

Test Case Creation:
The creation of test cases is scheduled to begin October 19, 2015. The deadline for creation of test cases for POC November 16, 2015. 
\ds{Grammar}
The deadline for final demonstration is February 1, 2016.
\ds{Grammar}
\newline
\newline
Test Case Implementation:
Implementing test cases for POC system test is scheduled to begin November 15, 2015. The implementation period is expected to last approximately one week.
\ds{Grammar}
\newline
\newline
Implementing test cases for Final Demonstration system test is scheduled to begin February 1, 2016. The implementation period is expected to last approximately three weeks. 
\ds{Grammar}


\subsection{Budget}
N/A

%%%%%%%%%%%%%%%%%%%%%%%%
%
%	3.) Software Specification
%
%%%%%%%%%%%%%%%%%%%%%%%%

\section{ Software Specification}
This section provides; the functional requirements the 
software is expected to complete, and the non-functional requirements the software is expected to exhibit.

%3.1 Functional Requirements
\subsection{Functional Requirements}

\noindent
\begin{itemize}
\item The product shall scan a bar code to access the restaurant menu from a database
\item The product shall allow the user to register an account for complete access to the products 
\ds{Multiple products or one? Should be ``products' " or ``product's" respectively.}
payment services
\item The product shall allow the user to report bugs
\item The product will allow users to place their order
\item The product will allow users to pay for their order using a credit card
\end{itemize}  

%3.2 Nonfunctional Requirements
\subsection{Nonfunctional Requirements}
Priority nonfunctional requirements are performance, ease of use, reliability, security and maintainability.


%%%%%%%%%%%%%%%%%%%%%%%%
%
%	4.) Test Types
%
%%%%%%%%%%%%%%%%%%%%%%%%

\section{Evaluation}

The following section first presents methods and constraints used during the evaluation process. This is followed by extent of testing and test tools to be used.
\ds{The last line is awkward.}

%4.1 Methods and Constraints
\subsection{ Methods and Constraints} 

%4.1.1 Methodology
\subsubsection{Methodology} 
Testing of Smart Waiter will consist of various types of tests explained in the next section. Manual testing will be conducted for the most part, some automation testing will be in place.

% 4.1.2 Extent of Testing
\subsubsection{Extent of Testing}
The extent of testing will be end to end for the final demonstration. For POC, specific system tests will be conducted.

% 4.1.3 Test Tools
\subsubsection{Test Tools}
To avoid regressions and catch bugs quickly two main test tools will be used. Namely the following;
\begin{itemize}
\item JUnit \textemdash Junit 
\ds{``JUnit"}
will be used to build an automation test suite for the application. Specifically, JUnit is a testing framework used to implement unit testing in Java. 
This tool will easily integrate into the android studio build environment used for application.
\ds{Awkward. I'm not sure what you're saying.}
In addition, it will help developers leverage their Java knowledge used during application development. 
\item CodeReviewer \textemdash CodeReviewer will be used for code review process mandatory for all developers. This tool helps post code reviews and provides features such as annotations, diff viewers, easy communication and feedback. This tool will be used for static tests to produce high quality code with less defects
 \end{itemize}

% 4.1.4 Testing Constraints
\subsubsection{ Testing Constraints}
N/A

\subsection{Types of Tests}

\subsubsection{Functional Testing}
Functional testing is in place to uncover errors that occur in implementing requirements or design specifications. Concentration is on results rather then the internals functions of the program. This type of testing is very effective to evaluate functional requirements.

\subsubsection{Structural Testing}
Structural testing is in place to uncover errors during implementation of the application. Concentration is on evaluating structure of the application. This type of testing focuses on non functional requirements.  

\subsubsection{Unit Testing}
Unit testing refers to providing specific input to a system and verifying it evaluates to expected output.This process can be done manually or automated.

\subsubsection{Manual and Automatic Testing}
Manual testing is done by people, while automatic is processed through scripts. 

\subsubsection{Static Testing}
Static testing refers to testing techniques that simulate a dynamic environment. This does not involve program execution. Instead, a walk-though through
\ds{Typo}
will be performed checking pre and post conditions evaluate to requirements specified. As well to make sure proper syntax is used thoroughly.This type of testing is crucial in design stage. \\
Static testing will be enforced through out the development process. A code review process will be put into place to strictly manage the quality of the code. Specifically, developers will have to sign off on the quality of the code before it gets checked into the production repo.
\ds{``repository"}
The tool used for code reviews be Codereviewer. \\\\
Below we have highlighted the code review process;\\
\includegraphics[width=\textwidth,height=\textheight,keepaspectratio]{review_process.png}
\ds{This graphic would make more sense as a flowchart. Also, label your figures}
\\\\
The reviewers will assess the code quality and critique the code in following criteria;\\
\begin{itemize}
 \item General Unit Testing performed. Did it pass?
 \item Comment and Coding Conventions followed consistently?
 \item Error Handling done correctly?
 \item Resource Leaks within the code?
 \item Is the code thread safe?
 \item Is the code up to standard with performance?
 \item Does code provide correct functionality?
 \item Is the code secure?
 \end{itemize}
The reviewers will assess the code and mark issues found with severity. If any of the high severity issues found, then developer must fix code. For medium severity issues, developer can provide reasoning and try to convince reviewers. Low severity issues can possibly be dropped if they are nits. The reviewers will mark severity as follows;
\begin{itemize}
\item Naming Conventions and Coding style considered low severity
\item Control flow and Logical issues considered medium or high severity
\item Redundant Code considered medium or high severity
\item Performance Issues considered high severity
\item Security Issues considered high severity
\item Scalability Issues considered medium or high severity
\item Functional Issues considered high severity
\item Error Handling considered high severity
\item Reusability considered medium severity
\end{itemize}

\subsection{Dynamic Testing}
On the contrary, dynamic testing refers to executing the program while running test cases to view expected behaviour. This is done to find and fix defects in the program. This will be performed after implementation phase. All of the tests performed in the following sections are dynamic. Specifically, different techniques are used to perform this testing such as automated and manual test cases. However, due to the nature of the application all other testing will be dynamic since the values are all dynamic.

%%%%%%%%%%%%%%%%%%%%%%%%
%
%	5.) System Test Description
%
%%%%%%%%%%%%%%%%%%%%%%%%

\section{POC System Test Description}

%%%%%%%%%%%%%%%%%%%%%%%%%%%%%%%%%%%%%%%%
						%database design %
%%%%%%%%%%%%%%%%%%%%%%%%%%%%%%%%%%%%%%%%
\subsection{Database Testing}
Database testing is exclusively performed from structural testing point of view. The goal of this component of testing is to verify that the database queries always return valid and correct data, under all circumstances. Specifically, a separate database will be created for testing purposes, because any failed test case should not affect the state of the original database. 
\subsubsection{Test Factors}
For database testing, the following test factors will be considered;
\begin{itemize}
 \item Correctness
 \item Performance
 \item Security
 \item Reliability
 \end{itemize}
 
\subsubsection{Correctness}
In this section of database testing, tester must verify that the application receives complete and correct data from the couchbase server.
\ds{This is the first time I'm seeing the name ``couchbase" in your project.}
Automated unit tests will be written using JUnit to perform the following tests.
\subsubsection{Database Correctness Automated Tests }
\includegraphics[width=\textwidth,height=\textheight,keepaspectratio]{correctness_tests.png}

\subsubsection{Performance}
In this section of database testing, tester must verify that the data query is performed in a reasonable amount of time. The application should have a maximum loading period of 5 seconds to complete data queries. Moreover, considering the customer base and size of database our application, none of the following test cases should take more than 5 seconds.
\subsubsection{Database Performance Automated Tests }
\includegraphics[width=\textwidth,height=\textheight,keepaspectratio]{performance_tests.png}
\ds{5 seconds sounds like a really long time to query one item.}

\subsubsection{Security}
In this section of database query testing, tester must verify that database allows correct access control. Database penetration testing is not required because we will leverage couchbase security implementation, testing and certifications.
\subsubsection{Database Security Automated Tests }
\includegraphics[width=\textwidth,height=\textheight,keepaspectratio]{security_tests.png}
\ds{Graphic was not included properly. I've corrected it.}

\subsubsection{Reliability}
In this section of database testing, we will test the robustness and reliability of the database.  This type of testing is essential the health of the database system. There should be no failures or downtimes on the backend side. 
\subsubsection{Database Reliability Automated Tests }
\includegraphics[width=\textwidth,height=\textheight,keepaspectratio]{reliability_tests.png}

\section{Final Demonstration System Test Description}

\subsection{Barcode Scanning}
Smart-Waiter needs to ensure users are able to scan a barcode with minimal attempts. The number of expected attempts will be presented in the final SRS document.

\subsubsection{Test Type}
\begin{itemize}
  \item Manual
  \item Functional test
  \item Unit test
\end{itemize}

\subsubsection{Nonfunctional Test Factors}
\begin{itemize}
  \item Performance
  \item 	Correctness
  \item 	Ease of use
\end{itemize}

\begin{comment}
\subsubsection{Inputs}
\begin{itemize}
  \item Barcode to be scanned
\end{itemize}
\subsubsection{Outputs}
\begin{itemize}
  \item Restaurant menu queried from database
\end{itemize}
\subsubsection{Initial State}
\begin{itemize}
  \item Barcode scanning page in empty state
\end{itemize}
\end{comment}

\subsubsection{Methods of testing}
Dynamic testing is used to ensure correctness and data integrity, and to observe the application behaviour when given incorrect information.

\subsubsection{Test Cases}
\textbf{\textit{Functional Unit Test}}\newline
\newline
\textit{Summary}\newline
Manual black box tests will be performed to assess barcode scanning. This test will be in the form of unit testing. Various test cases described below will be conducted. This is effective as it replicates real world usage and will provide a census of expected number of successful attempts to evaluate nonfunctional test factors.
\newline
\includegraphics[width=\textwidth,height=\textheight,keepaspectratio]{barcode.png}\newline
\ds{Graphic was not included properly. I've fixed it}

\subsection{Account Login}
Smart-Waiter must use accounts to keep track of a user's personal information. The account module has to provide a secure login service. 
\subsubsection{Test Type}
\begin{itemize}
  \item Manual
  \item functional dynamic test
\end{itemize}
\subsubsection{Nonfunctional Test Factors}
\begin{itemize}
  \item Correctness
  \item data integrity
\end{itemize}

\begin{comment}
\subsubsection{Inputs}
\begin{itemize}
  \item A new user's credentials, with valid information: username, first name, last name, date of birth, email, address, credit card
  \item A new user's credentials, with valid information but a fake credit card
  \item A Google account
  \item A Facebook account
\end{itemize}
\subsubsection{Outputs}
\begin{itemize}
  \item A message M, containing either a success of failure message, depending on the account information given. 
  \item A new user's credentials, with valid information but a fake credit card
  \item My Account menu
  \item Add Credit Card menu
  \end{itemize}
\subsubsection{Initial State}
Create account menu, empty
\end{comment}

\subsubsection{Methods of testing}
Dynamic testing is used to ensure correctness and data integrity, and to observe the application behaviour when given incorrect information.
\subsubsection{Test Cases}
\textbf{\textit{Dynamic Testing}}\newline
\newline
\textit{Summary}\newline
Manual dynamic tests will be performed to evaluate the account creation module. This test will be in the form of structural testing. The tests being performed on the module are presented below. 
\newline
\includegraphics[width=\textwidth,height=\textheight,keepaspectratio]{accountTC.png}
\ds{Test 1 should state it uses a username that is already in use. Why are
there no tests for invalid Google/Facebook accounts?}

\subsection{Order Transaction}
Smart-Waiter needs to ensure that a user can send in their order, and pay for their order easily and securely. Order transaction will be vigorously tested to ensure complete customer satisfaction.  
\subsubsection{Test Type}
\begin{itemize}
  \item Manual 
  \item functional dynamic test 
 \end{itemize} 
\subsubsection{Nonfunctional Test Factors}
\begin{itemize}
  \item Correctness 
  \item Reliability 
  \item Data integrity 
  \item Data security 
  \item Ease of use
 \end{itemize} 

\begin{comment}
\subsubsection{Inputs}
A set \texorpdfstring{n\textsubscript{i}}{ni} of 10 random orders consisting of different menu items, created manually, where i = 0, 1, 2 .. 9, where i = 0..4 are invalid orders with respect to the restaurant's policies and where i = 5..9 are valid orders. \\
A valid credit card \\
An expired credit card \\
A fake credit card \\
A VISA debit card \\
A VISA gift card \\
\subsubsection{Outputs}
An order summary \texorpdfstring{O\textsubscript{i}}{Oi}, where i corresponds to the \texorpdfstring{i\textsuperscript{th}}{ith} order from the set of orders \texorpdfstring{n\textsubscript{i}}{ni}. \\
A message M, containing either a success or failure message, depending on the card type used. 

\subsubsection{Initial State}
Order Test Cases: Restaurant menu module
Credit Card Test Cases: Payment confirmation menu
\end{comment}

\subsubsection{Methods of testing}
Dynamic testing is used to ensure validity, record the number of successful tests given a sample.
\subsubsection{Test Cases}
\textbf{\textit{Dynamic Testing}}\newline
\newline
\textit{Summary}\newline
Manual dynamic tests will be performed to evaluate the order transaction module. This test will be in the form of structural testing. It will test vigorously for all possible inputs of credit card information and account details. The tests being performed on the module are presented below. 
\newline
\includegraphics[width=\textwidth,height=\textheight,keepaspectratio]{orderTransactionTC.png}
\ds{What are these inputs for your test case 1 and 2? Where did the `n's come from?}
\subsection{Usability Testing}
The goal of usability testing is to verify if a user can successfully use all functionalities of the application from start to finish in a timely manner. This type of testing aids in evaluating if functional requirements are met. As well it provides scope for evaluating non functional requirements.
\subsubsection{Nonfunctional Test Factors}
\begin{itemize}
  \item Reliability 
  \item Performance
  \item 	Correctness
  \item 	Ease of use
\end{itemize}

\begin{comment}
\subsubsection{Inputs}
Input will consist of user input for entire app:
\begin{itemize}
  \item Account login
  \item Barcode
  \item Menu items selected
  \item Credit card details
\end{itemize}
\subsubsection{Outputs}
Output will consist of a response for each user input:
\begin{itemize}
  \item Successful login in response to account login
  \item Restaurant menu in response to barcode scan
  \item Order confirmation in response to menu items selected
  \item Transaction confirmation in response to credit card transaction
\end{itemize}
\subsubsection{Initial State}
\begin{itemize}
  \item User starts at the beginning of the process; account login page
\end{itemize}
\end{comment}

\subsubsection{Methods of testing}
Dynamic testing is used to evaluate end to end system testing. This effectively evaluates functional non functional requirements entirely of the system. 
\subsubsection{Test Cases}
\textbf{\textit{Functional Unit Test}}\newline
\newline
\textit{Summary}\newline
Manual black box tests will be performed to assess usability of the application. This test will be in the form of unit testing. Six participants will be selected to evaluate this test. At least three will have some experience with using android applications. The remaining will have little to no experience. Each participant will be asked to conduct a walk through of the application. Number of errors made will be counted and recorded to further evaluate functional requirements and improve test cases. As well, qualitative feedback from participants will be taken in the end. For example, participants will be asked to comment on things they liked and things they didn't like. Also, they will be asked to provide a rating out of 5 in terms of experience; 1 being a poor experience and 5 being a wonderful experience. This feedback will help evaluate non function requirements; user interface appearance, learning requirements and accessibility. 
\ds{If you are surveying users, you should include the questions you will be asking.
This can be in the main document or as an appendix.}
\newline

\subsubsection{Survey Questions}
The survey consists of seven questions. Participants are asked to give each question a score between 1 to 5. 1 being strongly disagree and 5 being strongly agree.
\begin{enumerate}
  \item I was able to complete the task quickly using the system
  \item It was easy to learn how to use the system
  \item I prefer using Smart-Waiter over traditional sense
  \item The system has all functions and capabilities i expect it to have
  \item The interface of the system was pleasant
 \item Whenever I made a mistake in the system, I could recover easily and quickly
 \item Overall I was happy with the system
\end{enumerate}

\section{Testing Schedule}
Active testing is necessary to build Smart-Waiter into a successful app.
\ds{``application"}
Therefore, testing will be performed on numerous occasions throughout the development of this application. 
\ds{When specifically? This should be in your schedule.}
Individual testing will be performed before any code is committed to the repository.
\ds{That is fine, however, there should be milestones where testing will be performed 
on a predetermined schedule.}
Furthermore, as newer features and functions are being added and the development of Smart-Waiter progresses, the testing team and developers will be changing the test cases accordingly. 
\ds{This last bit is unnecessary, as your test plan will be updated as you
update your test cases.}
\newline
\ds{Who will be testing what and when?}
\newline
\includegraphics[width=\textwidth,height=\textheight,keepaspectratio]{TestingSchedule.png}

\ds{I did not mark all of your spelling and grammar mistakes. You should go 
through the document and find them on your own. There are also many more possibilities
for test cases that you have not yet considered.}

\end{document}
