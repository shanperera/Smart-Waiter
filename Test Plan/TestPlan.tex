\documentclass[12pt]{article}

\usepackage{bm}
\usepackage{amsmath}
\usepackage{amsfonts}
\usepackage{amssymb}
\usepackage{graphicx}
\usepackage{colortbl}
\usepackage{xr}
\usepackage{hyperref}
\usepackage{longtable}
\usepackage{xfrac}
\usepackage{tabularx}
\usepackage{float}
\usepackage{siunitx}
\usepackage{booktabs}

%\usepackage{refcheck}

\hypersetup{
    bookmarks=true,         % show bookmarks bar?
      colorlinks=true,       % false: boxed links; true: colored links
    linkcolor=red,          % color of internal links (change box color with linkbordercolor)
    citecolor=green,        % color of links to bibliography
    filecolor=magenta,      % color of file links
    urlcolor=cyan           % color of external links
}
\newcommand{\wss}[1]{\authornote{magenta}{SS}{#1}}
\newcommand{\ds}[1]{\authornote{blue}{DS}{#1}}


\newcommand{\tclad}{T_\text{CL}}
\newcommand{\degree}{\ensuremath{^\circ}}
\newcommand{\progname}{SWHS}

\usepackage{fullpage}

\begin{document}

\title{Verification and Validation Plan for Solar Water Heating Systems Incorporating 
Phase Change Material} 
\author{Maya Grab}
\date{\today}
	
\maketitle

\tableofcontents

%%%%%%%%%%%%%%%%%%%%%%%%
%
%	1.) General Information 
%
%%%%%%%%%%%%%%%%%%%%%%%%

\section{General Information}
The following section provides an overview of the Verification and Validation (V\&V) Plan 
for a solar water heating systems incorporating phase change material simulator.
 This section explains the purpose of this document, the scope of the system,
  common definitions, acronyms and abbreviations that are used in the document,
   and an overview of the following sections

%1.1 Purpose
\subsection{Purpose}
The main purpose of this document is to describe the verification and validation 
process that will be used to test a simulation for solar water heating systems incorporating PCM.
This document is indented to be used as a reference for all future testing and will
be used to increase confidence in the software implementation.  

This document will be used as a starting point for the verification and validation report. The 
test cases presented within this document will be executed and the output will be analyzed to 
determine if the software is implemented correctly.  


%1.2 Scope
\subsection{Scope}


%1.3  Definitions, Acronyms, and abbreviations 
\subsection{Acronyms, Abbreviations, and Symbols }

\renewcommand{\arraystretch}{1.2}
\begin{tabular}{l l} 
  \toprule		
  \textbf{symbol} & \textbf{description}\\
  \midrule 
  QA		&Quality assurance\\
  SRS		&Software requirements specification\\
  V\&V		& Verification and validation\\
  V\&VP 	& Verification and validation plan\\
  V\&VR 	& Verification and validation report\\
  PCM		& Phase change material\\
  SWHS		& Solar Water Heating System\\
  $\epsilon$& $10^{-2}$\\
  \bottomrule
\end{tabular}\\

%1.4 Overview of Document
\subsection{Overview of Document }
The following sections provide more detail about the V\&V of a solar water heating
 simulator. Information about the testing process is provided, and the software specifications
that were discussed in the SRS document are stated.  The evaluation process that will be followed during 
testing is outlined, and test cases for both the system testing and unit testing are provided 

%%%%%%%%%%%%%%%%%%%%%%%%
%
%	2.) Plan
%
%%%%%%%%%%%%%%%%%%%%%%%%

\section{Plan}
This section provides a description of the software that is being tested, the team that will
perform the testing, the milestones for the testing phase, and the budget allocated to the testing. 

%2.1 Software Description
\subsection{Software Description}
The software being tested is a simulator for a SWHS
incorporating PCM. Given the physical parameters of the system,
 including dimensions, properties of the water and PCM,and relevant physical constants,
  the simulator calculates the changes in temperature and energy of the water and PCM 
  over time.

%2.2 Test Team
\subsection{Test Team} 
The team that will execute the test cases, write and review the V\&VR consists of:

\begin{itemize}
 \item Maya Grab 
 \item Dr.\ Spencer Smith
 \item Thulasi Jegatheesan 
\end{itemize}  

%2.3 Milestones
\subsection{Milestones}

%2.3.1 Location
\subsubsection{Location}
The location where the testing will be performed is Hamilton Ontario. The institution that
will be performing the testing is McMaster University. 


%2.3.1 Dates and Deadlines
\subsubsection{Dates and Deadlines}
Test Case:
~\newline
The creation of the test cases for both system testing and unit testing is 
scheduled to begin June $1^\text{st}$ 2015.
The deadline for the creation of the test cases is June 15th 2015. 
~\newline
~\newline
Test Case Implementation:
~\newline
Implementing code for the automation of the unit testing is scheduled to being 
June 15th 2015. The implementation period
is expected to last approximately two weeks and has a deadline of June 30th 2015.
~\newline
~\newline
Verification and Validation Report:
~\newline
The writting of the V\&VR is scheduled to begin July 1st 2015 and end on July 15th 2015. 

%2.4 Budget
\subsection{Budget}
The budget for the testing of this system is being funded by McMaster University and NSERC.

%%%%%%%%%%%%%%%%%%%%%%%%
%
%	3.) Software Specification
%
%%%%%%%%%%%%%%%%%%%%%%%%

\section{ Software Specification}
This section provides the functional requirements, the business tasks that the
software is expected to complete, and the nonfunctional requirements, the
qualities that the software is expected to exhibit.

%3.1 Functional Requirements
\subsection{Functional Requirements}

\noindent
\begin{itemize}
\item Input the physical constants, properties and initial temperatures of water
 and PCM, and dimensions of the tank  
\item Verify that the inputs satisfy the required physical constraints 
\item Compute the calculated values required to solve the governing differential equations
\item Calculate the temperatures and energy of water and PCM over time.
\end{itemize} 

%3.2 Nonfunctional Requirements
\subsection{Nonfunctional Requirements}
Priority nonfunctional requirements are correctness, understandability,
reliability, and maintainability.


%%%%%%%%%%%%%%%%%%%%%%%%
%
%	4.) Test Types
%
%%%%%%%%%%%%%%%%%%%%%%%%

\section{Evaluation}
This section first presents the methods and constraints that are to be used
during the evaluation process. This is followed by how the data obtained by the
testing will be evaluated, which includes: how the data will be recorded, how to
move from one test to the next, and how to determine if the test was successful.

%4.1 Methods and Constraints
\subsection{ Methods and Constraints} 

%4.1.1 Methodology
\subsubsection{Methodology} 

% 4.1.2 Extent of Testing
\subsubsection{Extent of Testing}

% 4.1.3 Test Tools
\subsubsection{Test Tools}


% 4.1.4 Testing Constraints
\subsubsection{ Testing Constraints}

\subsection{Types of Tests}

\subsubsection{Functional Testing}

\subsubsection{Structural Testing}

\subsubsection{Unit Testing}

\subsubsection{Manual and Automatic Testing}

\subsubsection{Static and Dynamic Testing}

%%%%%%%%%%%%%%%%%%%%%%%%
%
%	5.) System Test Description
%
%%%%%%%%%%%%%%%%%%%%%%%%

\section{System Test Description}


%5.x Test identifier 
\subsection{POC Test}

% 5.x.2 Input
\subsubsection{Barcode Scanning}

\subsubsection{Database Querying}

\subsection{Remaining Test}

\subsubsection{Account Login}

\subsubsection{Order Transaction}

\subsubsection{Usability Testing}

\subsubsection{Performance Testing}

\section{Testing Schedules}

\section{Automated Testing Plan}


\end{document}
